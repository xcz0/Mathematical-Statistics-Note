\chapter{描述性统计}

\section{单变量样本}

设$x = \{x_1,\dots,x_n\}$ 是单变量样本 $X_1,X_2,\cdots,X_n$ 的一个实现。这些随机变量独立同分布。

一个分布通常有两个重要的数值特征:位置(location)和散度(dispersion)。位置通常由均值和中位数描述。若分布是对称的,则两者等价;若分布右偏(左偏),则均值大于(小于)中位数。对于一个未知的分布,通常使用样本均值和样本中位数描述其位置。

\begin{definition}[样本均值]\label{def:sample mean}
    将样本 $X_1,X_2,\cdots,X_n$ 的一个统计量 
    \[ \overline{X} = \frac1n\sum_{i=1}^{n}X_{i}. \]
    称为\textbf{样本均值}(sample mean)。
\end{definition}

\begin{definition}[样本均值]\label{def:sample median}
    将样本 $X_1,X_2,\cdots,X_n$ 的一个统计量 
    \[ M = \operatorname{med}(x) = \begin{cases}
        x_{(\frac{n+1}{2})} & n \text{为奇数} \\
        \frac{x_{(\frac{n}{2})} + x_{(\frac{n}{2}+1)}}{2} & n \text{为偶数}
    \end{cases} \]
    称为\textbf{样本中位数}(sample median)。其中顺序统计量$x_{(1)}\leq\dots\leq x_{(n)}$由$x_1,\dots,x_n$排列得到。
\end{definition}

分布的散度则对应着方差(或标准差)和四分位数(interquartile range, IQR)。


\begin{description}
      \item [样本均值](sample mean)
            \[ \bar{x} = \frac1n\sum_{i=1}^{n}x_{i}. \]
      \item [样本方差](sample variance)
            \[ s^{2} = \frac{1}{n-1}\sum_{i=1}^{n}(x_{i}-\bar{x})^{2}. \]
            由此可得\textbf{样本标准差}(sample standard deviation) $s = \sqrt{s^2}$.
      \item [样本中位数](sample median)
            \[ M = \operatorname{med}(x) = \begin{cases}
                        \quad x_{(k)},                  & n = 2k-1 \\
                        \frac{1}{2}(x_{(k)}+x_{(k+1)}), & n = 2k
                  \end{cases} \]
            
      \item [四分位数](quartile)
            \[ Q_{1} = \operatorname{med}(x\cap(-\infty,M)), \qquad
                  Q_{3} = \operatorname{med}(x\cap(M,+\infty)). \]
      \item [四分位距](inter quartile range)
            \[ \mathit{IQR} = Q_{3} - Q_{1}. \]
      \item [极差](range)
            \[ x_{(n)}-x_{(1)} = \max(x)-\min(x) = \max_{1\leq i,i' \leq n}\{x_{i}-x_{i'}\}. \]
\end{description}

\begin{definition}[经验分布函数]
      样本 $X_1,X_2,\cdots,X_n$ 的\textbf{经验分布函数}为:
      \[ F_n(x) = \frac1n\sum_{i=1}^n \1{X_i \le x}  \]
\end{definition}

\subsection{直方图}

\subsection{箱线图}

\begin{definition}
      \textbf{五数概括法}(five-number summary): 以$x_{(1)},Q_{1},M,Q_{3},x_{(n)}$总结$x$, 可用\textbf{箱线图}(boxplot)表示.
\end{definition}

\subsection{QQ图}



\section{相关性}