\chapter{假设检验}

\section{假设检验的概念}

\subsection{基本思想}

\textbf{基本思想}:将样本空间$\Omega$拆分为两个不相交的集合$\Omega_0, \Omega_A$即
\[ \Omega_0 \cap \Omega_A=\emptyset,\ \Omega_0 \cup \Omega_A=\Omega \]
接下来通过数据选择应该接受两个假设
\[ H_0:\Theta \in \Omega_0,\ H_A:\Theta \in \Omega_A \]
中的哪一个。

\textbf{问}:是否能用点估计代替假设检验?即若估计结果$\hat{\Theta} \in \Omega_0$,则接受假设$H_0$;或反之。

假设某一总体遵循二项分布$B(n,p)$,做出如下假设:
\[ H_0:p=0.5,\ H_A : p \neq 0.5 \]
若$n=10^5$,估计值$\hat{p}=0.50001$。按估计结果应认为$H_A$成立,然而一般按情理而言,似乎$H_0$更恰当。这体现出点估计与假设检验的不同:点估计的结果是客观的;而假设检验的结果包含一定的主观性。

若假设可用一个参数的集合表示,该假设检验问题称为\textbf{参数假设检验}问题,否则称为\textbf{非参数假设检验}问题。上例就是一个参数假设检验问题,而对假设“总体为正态分布”作出检验的问题则是一个非参数假设检验问题。

\begin{definition}[原假设与对立假设]
    设有来自某一个参数分布族$\{ F(x,\theta) | \theta \in \Theta \}$的样本$x_1,\cdots ,x_n$,其中$\Omega$为参数空间。设$\Theta_0 \in \Theta,\ \Theta_0 \neq \emptyset$,则命题$ H_0:\Theta \in \Omega_0$称为或\textbf{原假设}或零假设(null hypothesis);并称命题$H_A:\Theta \in \Omega_A,\ \Theta_A=\Theta-\Theta_0$为$H_0$的\textbf{对立假设}或备择假设(alternative hypothesis)。
\end{definition}

\begin{definition}
    如果假设(原假设或对立假设)只含一个点,则称之为\textbf{简单假设}(simple hypothesis),否则称为\textbf{复合假设}(composite hypothesis)。
\end{definition}

当$H_0$为简单假设时,其形式可写成$H_0:\theta \theta_0$,此时的备择假设通常有如下三种可能:
\[ H_A':\theta \neq \theta_0,\ H_A'':\theta < \theta_0,\ H_A''':\theta > \theta_0,\ \]
称$H_0 \vs H_A'$为双侧假设或双边假设;$H_0 \vs H_A''$以及$H_0 \vs H_A'''$为单侧假设或单边假设。

\subsection{Neyman-Pearson范式}

\begin{definition}
    对于样本空间$S$,若有一种规则将其划分为两个互不相交的区域$W,\overline{W}$,并且当样本$\mathbf{x} \in W$时,选择假设$H_A$,否则接受假设$H_0$,则将这种划分称为\textbf{检定}(test),将$W$与$\overline{W}$分别称为\textbf{拒绝域}(rejection region, RR)与\textbf{接受域}(acceptance region, AR)。
\end{definition}

根据检定结果与实际参数的不同,参见表\ref{table:test_4_type}有4种可能的情况:
\begin{table}[!htp]
    \centering
    \caption{检验的4种情况}\label{table:test_4_type}
    \begin{tabular}{ccc}
                                                    & \multicolumn{2}{c}{总体情况}                   \\ \cline{2-3}
        \multicolumn{1}{c|}{数据情况}               & $H_{0}$为真                  & $H_{1}$为真     \\ \hline
        \multicolumn{1}{c|}{$\mathbf{x} \in W$}     & {\red 一类错误}              & 正确            \\ \hline
        \multicolumn{1}{c|}{$\mathbf{x} \in W^{c}$} & 正确                         & {\red 二类错误} \\ \hline
    \end{tabular}
\end{table}



\section{正态总体参数假设检验}

\section{然似比检验}

\section{一致最优检验}

\section{无偏检验}

\section{假设检验与区间估计}

\begin{problemset}[错题记录]
    \item (茆7.1.1)
    \item
\end{problemset}
