\chapter{参数估计}

\section{点估计的概念}

\begin{definition}[点估计]
    用于估计参数$\gamma = g(\theta)$的统计量$T(\mathbf{X})$称为\textbf{估计量}(estimator), 记为$\hat{\gamma} = T(X)$. 估计式是随机变量, 若给定统计模型, 则其分布由参数$\theta$决定, 将观测数值代入估计式后得到的值称为\textbf{估计值}.
\end{definition}

将数据代入估计式得到的估计值并不是真实的参数值,若使用新的样本,很可能会得到不同的估计值。

\begin{definition}[标准误差]
    估计式的\textbf{标准误差}(standard error)定义为其抽样分布的标准差, 即:
    \[ \sqrt{\operatorname{Var}_{\theta}(\hat{\theta})} \]
\end{definition}

\begin{definition}[无偏估计]
    估计式的\textbf{偏差}(standard error)定义为其抽样分布的均值与实际参数的偏差, 即:
    \[ E_{\theta}(\hat{\theta})-\theta \]
    若其为零, 则称次估计为\textbf{无偏估计}
\end{definition}

\begin{definition}[相合]
    若当样本容量$n\to\infty$时, 有
    \[ \hat{\gamma} \xrightarrow{\P} \gamma, \quad \forall \theta\in\Theta. \]
    则称$\hat{\gamma}$是\textbf{相合的/一致的}(consistent), 如果上式中$\xrightarrow{\P}$可以增强为$\xrightarrow{\as}$, 则称$\hat{\gamma}$是\textbf{强相合的}.
\end{definition}

\section{矩估计}

\begin{definition}[总体矩与样本矩]
    若$X_1,\cdots ,X_n \overset{\text{i.i.d.}}{\sim} P_{\theta}$,记$k$阶\textbf{总体矩}为:
    \[ \mu_k(\theta) = \E_{\theta}[X_i^k] ,\quad k \in \mathbb{N}\]
    记$k$阶\textbf{样本矩}为:
    \[ \hat{\mu_k}=\frac{1}{n}\sum_{i=1}^n X_i^k ,\quad k \in \mathbb{N}\]
\end{definition}

\begin{proposition}
    $k$阶样本矩是关于总体分布$k$阶矩的无偏估计。
\end{proposition}
\begin{proof}

\end{proof}

\begin{definition}
    参数$\gamma = g(\theta)$的\textbf{矩{\color{lightgray}(方法)}估计量}(method of moments estimator)定义为
    \[ \hat{\gamma}^{\mathrm{MoM}} = g(\hat{\theta}^{\mathrm{MoM}}) , \]
    其中$\hat{\theta}^{\mathrm{MoM}}$对选定的$k$满足
    \[ \mu_{k}(\hat{\theta}^{\mathrm{MoM}}) = \hat{\mu}_{k} . \]
\end{definition}


矩方法的步骤:
\begin{enumerate}
    \item 将低阶矩写为参数的函数,一般阶数与参数个数相同$(\mu_i)_n=f((\theta)_n)$;
    \item 找出上一步骤的反函数,通过矩表达参数$(\theta)_n=f^{-1}((\mu_i)_n)$;
    \item 将样本矩代入,得到参数的估计式$\hat{\theta}=f^{-1}(\hat{\mu_i})$
\end{enumerate}

\begin{example}[泊松分布的矩估计]\label{moment_estimate_Poisson}
    设$X_1, \cdots ,X_n \overset{\text{i.i.d.}}{\sim} P(\lambda)$,则其一阶矩$\mu_1=\lambda$,所以$\lambda=\mu_1$,其矩估计为:
    \[ \hat{\lambda} = \hat{\mu_1} = \overline{X} \]
\end{example}

\begin{example}
    设$X_{1},\dots,X_{n}$是独立同分布的连续型随机变量, 具有p.d.f.
    \[ f_{\lambda,a}(x) = \lambda\mathrm{e}^{-\lambda(x-a)}\1_{[x>a]}, \quad x\in\R, \]
    其中$\lambda > 0$和$a \in \R$未知(\emph{注}: 相应的统计模型是带有\emph{位置}(location)参数和\emph{速率}(rate)参数的指数分布). 易见$X_{i} \stackrel{d}{=} a + Y/\lambda$, 其中$Y \sim \mathrm{Exponential}(1)$. 利用$\E[Y]=1$和$\E[Y^{2}] = 2$, 可以得到
    \[ \mu_{1}(\lambda,a) = a + 1/\lambda, \quad\&\quad
        \mu_{2}(\lambda,a) = a^{2} + 2a/\lambda + 2/\lambda^{2} . \]
    方程
    \[ \mu_{k}(\hat{\lambda}^{\mathrm{MoM}},\hat{a}^{\mathrm{MoM}}) = \hat{\mu}_{k}, \quad k = 1,2 \]
    的解
    \[ \hat{\lambda}^{\mathrm{MoM}} = 1\Big/\sqrt{\hat{\mu}_{2}-\hat{\mu}_{1}^{2}}, \quad\&\quad
        \hat{a}^{\mathrm{MoM}} = \hat{\mu}_{1}-\sqrt{\hat{\mu}_{2}-\hat{\mu}_{1}^{2}} \]
    即为$(\lambda,a)$的一种矩估计.
\end{example}

\begin{remark}
    矩方法得到的估计量往往可以援引大数定律(LLN)来说明\emph{相合性}. 
\end{remark}

\section{极大然似估计}

极大然似估计的基本思想:对于参数空间中的每一个参数,计算在此参数下,观测数据的发生概率,选取最大概率对应的参数。

\begin{definition}[然似函数]
    设随机变量$X_1,\cdots ,X_n$的联合密度函数(或质量函数)为$f(x_1,\cdots ,x_n|\theta)$。对于某一组观测数据$x_1^* ,\cdots ,x_n^*$,其\textbf{然似函数}(likelihood function)为:
    \[ \mathfrak{L}(\theta) = f(x_1^* ,\cdots ,x_n^*|\theta) \]
    对数然似函数(log likelihood function)为$l(\theta)=\log\mathfrak{L}(\theta)$
\end{definition}
\begin{remark}
    联合密度质量函数代表概率,但联合密度函数不是,代表概率所占比例。然似函数是关于参数$\theta$的函数,不是概率,对整个参数空间的积分未必等于一。
\end{remark}

\begin{definition}[极大然似估计]
    可使然似函数取最大值的参数被称为\textbf{极大然似估计},即
    \[ \hat{\theta}=\max_{\theta \in \Theta}\mathfrak{L}(\theta) \]
\end{definition}
\begin{remark}
    由于对数函数单调递增,最大化然似函数等价于最大化对数然似函数。
\end{remark}

若样本来源变量独立同分布,即$f(x_1,\cdots ,x_n|\theta)=\prod_{i=1}^nf(x_i|\theta)$,则其然似函数和对数然似函数分布可写为:
\[ \mathfrak{l}(\theta) = \prod_{i=1}^nf(x_i^*|\theta), \quad l(\theta) = \sum_{i=1}^n \log f(x_i^*|\theta) \]

\begin{proposition}[极大然似估计的不变性]
    设$\hat{\theta}$是参数$\theta$的极大然似估计,那么对于参数$\theta$的任意函数$\gamma = g(\theta)$,其极大然似估计$\hat{\gamma}=g(\hat{\theta})$
\end{proposition}
\begin{proof}
    %TODO
\end{proof}

\section{矩估计与极大然似估计的渐进性质}



\section{一致最小方差无偏估计}

\section{Cramer-Rao不等式}

\section{概率密度函数的核估计}
