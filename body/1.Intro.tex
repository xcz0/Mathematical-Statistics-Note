\chapter{导论: 从数据中学习}\label{ch:intro}

\section{统计模型}\label{sec:inference}

统计学的主要研究内容是如何处理\underline{随机数据}。研究者希望设计试验收集数据,从中获取信息,并借此\underline{解释}和\underline{预测}数据。统计推断(inference)基于概率模型,对两个基本问题:\underline{估计}(estimation)与\underline{检验}(testing)进行回答。

\begin{definition}[样本]
      在实验中获取的观测值$\mathbf{x}=(x_1, \cdots ,x_n)$视为某个概率空间$(\Omega,\mathscr{F},\P)$上的随机元$\mathbf{X}=(X_1,\cdots ,X_n)$的\underline{实现}(realization)。一组观测值构成一个\textbf{样本}(sample),数据的数目$n$称为\textbf{样本容量}(sample size)。
\end{definition}

\begin{definition}[总体]
      假设$X_i$独立同分布,其分布$P(\bullet):=\P\{X_i \in \bullet\}$称为\textbf{总体}(population)。
\end{definition}

在统计学中可以认为总体$P$包含我们想知道的一切信息,然而(至少部分)是未知的。我们希望用样本$\mathbf{x}$推断总体$P$的性质。

\begin{definition}[参数]
      在总体中固定但未知的常数称为\textbf{参数}, 记为$\theta$. 参数所有可能的取值构成\textbf{参数空间}, 记为$\Theta$, 可以是有限维或者无限维。
\end{definition}
\begin{remark}
      参数的函数同样固定且未知,故也视为参数. 
\end{remark}

\begin{definition}[统计模型]
      \textbf{统计模型}(statistical model)是样本$\mathbf{X}$对应的所有可能的总体$P$的集合,记为$\mathcal{P}$。在同一统计模型中,不同的总体通过参数区分,所以也记为:
      \[ \mathcal{P}=\{ P_{\theta}:\theta \in \Theta \} \]
\end{definition}

统计模型代表关于数据产生机制的先验(prior)知识。

\begin{definition}[可识别]
      若模型$\mathcal{P}$满足:
      \[ P_{\theta_1} \neq P_{\theta_2}, \forall \theta_1 \neq \theta_2 \]
      则称模型$\mathcal{P}$\textbf{可识别}(identifiable)
\end{definition}

\begin{definition}
      若$\Theta$有限维的,则称模型$\mathcal{P}$为\textbf{参数族}(parametric family);若参数空间$\Theta$是无限维的,则称为\textbf{非参数族}。
\end{definition}

我们常常只关心参数$\Theta$的某些分量的函数$\gamma=g(\theta)$,剩下的碍事且无用的部分称为\textbf{冗余参数}(nuisance parameter)。

\section{统计量}

\begin{definition}[统计量]
      给定样本$X_1,\cdots ,X_n \overset{\text{i.i.d.}}{\sim} P$,其中$P \in \mathcal{P}$是未知的总体. 若$ T:(\mathcal{X}_n,\mathscr{F}^n_\mathcal{X})\to (\mathcal{T} ,\mathscr{F}_\mathcal{T})$是已知的(不依赖$P$的)可测函数,则称$T(\mathbf{X})=T(X_1,\cdots ,X_n)$为\textbf{统计量}(statistic)。
\end{definition}

\begin{definition}
      统计量的分布称为它的称之为\textbf{抽样分布}(sampling distribution), 含有总体 (样本分布) 的一部分信息.
\end{definition}

统计量函数, 不依赖未知总体, 给出了一种数据约简(reduction). 对于实数值样本$x = \{x_{1},\dots,x_{n}\}$, 常见的统计量有: \label{descriptive-statistics}
\begin{description}
      \item [样本均值](sample mean)
            \[ \bar{x} = \frac{1}{n}\sum_{i=1}^{n}x_{i}. \]
      \item [样本方差](sample variance)
            \[ s^{2} = \frac{1}{n-1}\sum_{i=1}^{n}(x_{i}-\bar{x})^{2}. \]
            由此可得\textbf{样本标准差}(sample standard deviation) $s = \sqrt{s^2}$.
      \item [样本中位数](sample median)
            \[ M = \operatorname{med}(x) = \begin{cases}
                        \quad x_{(k)},                  & n = 2k-1 \\
                        \frac{1}{2}(x_{(k)}+x_{(k+1)}), & n = 2k
                  \end{cases} \]
            其中\emph{顺序统计量}$x_{(1)}\leq\dots\leq x_{(n)}$由$x_{1},\dots,x_{n}$排列得到.
      \item [四分位数](quartile)
            \[ Q_{1} = \operatorname{med}(x\cap(-\infty,M)), \qquad
                  Q_{3} = \operatorname{med}(x\cap(M,+\infty)). \]
      \item [四分位距](inter quartile range)
            \[ \mathit{IQR} = Q_{3} - Q_{1}. \]
      \item [极差](range)
            \[ x_{(n)}-x_{(1)} = \max(x)-\min(x) = \max_{1\leq i,i' \leq n}\{x_{i}-x_{i'}\}. \]
\end{description}

\begin{definition}
      \textbf{五数概括法}(five-number summary): 以$x_{(1)},Q_{1},M,Q_{3},x_{(n)}$总结$x$, 可用\textbf{箱形图}(boxplot)表示.
\end{definition}

\section{变量之间的关系}
考虑同一批实例的两个变量$x = (x_{i})_{1\leq i\leq n}$和$y = (y_{i})_{1\leq i\leq n}$, 我们或许认为$y$是值得关心的结果(outcome), 并猜想$x$对$y$造成了影响-----此时称$y$为\textbf{响应变量}(response variable), 称$x$为\textbf{解释变量}(explanatory variable).

常用的图示是\textbf{散点图}(scatterplot), 对每个实例$i$绘制数据点$(x_{i},y_{i})$. 我们往往期待\emph{线性}关系, 为此, 可以考虑对数据进行变换, 比如$\log : (0,\infty) \to \R$.

对于实值变量, 常用的统计量有:
\begin{itemize}
      \item \textbf{样本协方差}(sample covariance)
            \[ s_{xy} = \frac{1}{n-1}\sum_{i=1}^{n}(x_{i}-\bar{x})(y_{i}-\bar{y}). \]
      \item \textbf{样本相关系数}(sample correlation coefficient)
            \[ r_{xy} = \frac{s_{xy}}{s_{x}s_{y}} = \frac{1}{n-1}\sum_{i=1}^{n}\left(\frac{x_{i}-\bar{x}}{s_{x}}\right)\left(\frac{y_{i}-\bar{y}}{s_{y}}\right), \]
            其中$s_{x}$和$s_{y}$是相应的样本标准差.
\end{itemize}


