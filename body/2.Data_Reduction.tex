\chapter{抽样分布}

\section{统计量的极限分布}

\section{指数族}

\section{充分统计量}

若样本数据量大,可能难以解释。试验者希望提取样本值的一些关键特征以概括样本中的信息。这类数据简化(缩减)在计算统计学中通常以样本函数的形式实现,例如,样本均值、样本方差、最大观测值和最小观测值就是四个概括样本关键特征的统计量。

任意一个统计量$T(X)$都定义了一种数据简化方式。如果试验者只观测统计量$T(x)$而非整个样本$x$,则他必将满足$T(x) = T(y)$的$x$和$y$视作两个相同的样本,尽管事实可能并非如此。不同的统计量对数据中的信息划分有不同的方法。

依据某统计量简化样本数据可以看成样本空间$\mathcal{X}$上的一个划分。设$\mathcal{T}=\{t|\exists x \in \mathcal{X}, \text{s.t.} \quad  t=T(x) \}$为$\mathcal{X}$在$T(x)$下的象。则$A_t=\{x|T(x)=t, t \in \mathcal{T}\}$为$\mathcal{X}$若干划分。

原始数据包含了所有信息,规律或随机部分。若进行转化,则将丢失信息,可能有用,也可能无用。其中界限由假设的统计模型判断。转化后的可能结果:
\begin{enumerate}
	\item 留下部分有用信息:完备统计量
	\item 留下所有有用信息:充分统计量
	\item 不留下有用信息:辅助统计量
\end{enumerate}

\begin{example}
	设$X_1,\dots,X_n \thicksim Binomial(p) \quad \text{i.i.d.} $,设$T(X)=\sum X_i$。
	若$T=t$已知,则实验结果与$p$无关,由于:
	$$P(X|T)=\frac{P(X)}{P(T)}=\frac{p^t(1-p)^{n-t}}{\binom{n}{t} p^t(1-p)^{n-t}}=\frac{1}{\binom{n}{t}}$$
\end{example}

\begin{remark}
	$T \thicksim Binomial(n,p)$与$p$有关,而$X|T$与参数无关。即原始数据经$T$转化后的$T(X)$,仍包含所有关于参数的信息;而余下的$X|T$不再包含参数信息。
\end{remark}

\begin{definition}[充分统计量]
	假设样本$X$,满足分布$P(\theta)$,若统计量$S(X)$,$P(X|S)$与$\theta$无关,则称$S(X)$为关于$\theta$的充分统计量。
\end{definition}

\begin{remark}
	充分统计量的判断与统计模型有关,模型不当可能导致充分统计量实际不“充分”。
\end{remark}

\begin{theorem}[分解定理]
	$S(X)$为关于$\theta$的充分统计量的充要条件为:
	$$\exists g(),h() \text{s.t.} f(X|\theta)=g(S(X),\theta)h(X)$$
\end{theorem}

\begin{remark}
	直观理解:$P(X)=P(S)P(X|S)$
\end{remark}

\begin{proof}
	对于离散情况:\\
	充分:
	$$P(S=s)=\sum_{S(x)=s}P(X=x)=g(s,\theta)\sum_{S(x)=s}h(x)$$
	$$P(X=x|S=s)=\frac{P(X=x)}{P(S=s)}=\frac{h(x)}{\sum_{S(x)=s}h(x)}$$
	与$\theta$无关。\\
	必要:\\
	令$$g(s,\theta)=P(T=s|\theta),h(x)=P(X=x|S=s)$$即可
\end{proof}

\begin{theorem}
	若$S(X)$为关于$\theta$的充分统计量,则$\theta$的极⼤然似估计可表示为$S$的函数。
\end{theorem}

\begin{proof}
	然似函数为$g(S,\theta)h(X)$。由于$h(X)$为定值,故只需求$g(S,\theta)$的极值情况,故$\theta$的取值可由$S$的函数表示。
\end{proof}

为使数据尽可能精简,摈弃无用信息,定义极小充分统计量。

\begin{definition}[极小充分统计量]
	若统计量$M$满足
	$$\forall S, \exists h  \quad \text{s.t.} \quad
	 M=h(S) $$
	则其为关于$\theta$的充分统计量
\end{definition}

\section{完全统计量}

